\section{Conclusions}
\label{sec:conclusion}

While the stock markets of developed countries are still dominant in international markets, the influence of emerging economies has been growing recently. This growth has caused economists to suggest new economic blocs like MINT countries. Although MINT countries are in a growth process, they are quite open to being influenced by advanced economies such as the G7. In particular, emerging economies' stock markets are frequently highly sensitive and closely linked to financial and economic developments in the G7 countries due to the interlinked nature of financial markets. Consequently, it is imperative for investors and policymakers in both developed and emerging markets to comprehend the connections between them, especially in generating forecasts for stock price movement. For this purpose, the stock market indices of G7 countries and MINT countries, representing developed and emerging economies, respectively, were discussed and predicted in the study between 2012 and 2024. 

In the study, for the first time, a GNN technique, MTGNN, that allows consideration of spatio-temporal connections in multivariate time series was applied to predict the closing price of the main stock market indices of the economic blocs. Taking into account the countries' complex interconnections, which are not known in advance, in the forecasting process enhances prediction accuracy. It obtained stock price predictions with excellent accuracy via MTGNN, when compared to other AR, VAR-MLP, RNN-GRU, and TCN baseline methods.

Generating forecasts for stock price movements is advantageous for investors, investment managers, and policy-makers engaged in stock market prediction. New artificial intelligence-based deep-learning methods like GNN provide forecasts with unprecedented accuracy even in emerging economies like MINT, which have less stable economic environments and are more sensitive to domestic and global situations than those in developed countries. Future studies should be encouraged to employ these methods more frequently for better predictions in the economics and finance field.

