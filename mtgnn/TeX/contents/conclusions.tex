\section{Conclusions}
\label{sec:conclusion}
%
While the stock markets of developed countries continue to dominate
international markets, the influence of emerging economies has been growing
recently. This growth has led economists to propose new economic blocs, such as
the MINT countries. Although the MINT countries are in a growth phase, they
remain susceptible to influence from advanced economies like the G7. In
particular, the stock markets of emerging economies are often highly sensitive
and closely linked to financial and economic developments in G7 countries due to
the interconnected nature of financial markets. Consequently, it is crucial for
investors and policymakers in both developed and emerging markets to understand
these connections, especially when forecasting stock price movements. In this
vein, we predicted the prices of the stock market indices of G7 and MINT
countries by learning and taking into account the spatial relationships between
G7 and MINT countries between 2012 and 2024 with the help of graph neural
networks.

In the study, a graph neural network architecture called MTGNN was applied for
the first time to predict the prices of the main stock market indices of G7 and
MINT economic blocs. This architecture allows for the consideration of
spatio-temporal connections in multivariate time series. By accounting for the
complex interconnections between countries, which are not known in advance, the
forecasting process is enhanced, leading to improved prediction accuracy. The
MTGNN model achieved excellent accuracy in stock price predictions compared to
other baseline methods such as AR, VAR-MLP, RNN-GRU, and TCN.

Generating forecasts for stock price movements is beneficial for investors,
investment managers, and policymakers involved in stock market prediction. New
artificial intelligence-based deep learning methods, such as GNN, provide
forecasts with unprecedented accuracy, even in emerging economies like the MINT
countries, which have less stable economic environments and are more sensitive
to domestic and global situations than developed countries. Future studies
should be encouraged to employ these methods more frequently for better
predictions in the fields of economics and finance.
