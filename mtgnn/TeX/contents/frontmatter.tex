\begin{frontmatter}

%% Title, authors and addresses

%% use the tnoteref command within \title for footnotes;
%% use the tnotetext command for theassociated footnote;
%% use the fnref command within \author or \affiliation for footnotes;
%% use the fntext command for theassociated footnote;
%% use the corref command within \author for corresponding author footnotes;
%% use the cortext command for theassociated footnote;
%% use the ead command for the email address,
%% and the form \ead[url] for the home page:
%% \title{Title\tnoteref{label1}}
%% \tnotetext[label1]{}
%% \author{Name\corref{cor1}\fnref{label2}}
%% \ead{email address}
%% \ead[url]{home page}
%% \fntext[label2]{}
%% \cortext[cor1]{}
%% \affiliation{organization={},
%%            addressline={}, 
%%            city={},
%%            postcode={}, 
%%            state={},
%%            country={}}
%% \fntext[label3]{}

\title{Stock Market Telepathy: Graph Neural Networks Predicting the Secret Conversations between MINT and G7 Countries}

%% use optional labels to link authors explicitly to addresses:
%% \author[label1,label2]{}
%% \affiliation[label1]{organization={},
%%             addressline={},
%%             city={},
%%             postcode={},
%%             state={},
%%             country={}}
%%
%% \affiliation[label2]{organization={},
%%             addressline={},
%%             city={},
%%             postcode={},
%%             state={},
%%             country={}}

\author[label1,label2]{Nurbanu Bursa}

\affiliation[label1]{organization={Hacettepe University},%Department and Organization
            addressline={Department of Statistics}, 
            city={Beytepe},
            postcode={06800}, 
            state={Ankara},
            country={T\"{u}rkiye}}

\affiliation[label2]{organization={Boise State University},%Department and Organization
            addressline={Biomedical Research Institute}, 
            city={Boise},
            postcode={83725}, 
            state={Idaho},
            country={USA}}

\begin{abstract}
Emerging economies, particularly the MINT countries (Mexico, Indonesia, Nigeria, and T\"{u}rkiye), are gaining influence in global stock markets, although they remain susceptible to the economic conditions of developed countries like the G7 (Canada, France, Germany, Italy, Japan, the United Kingdom, and the United States). The interconnectedness of financial markets makes understanding these relationships crucial for investors and policymakers to predict stock price movements accurately. This study examined the main stock market indices of G7 and MINT countries from 2012 to 2024, using a recent Graph Neural Network (GNN) technique called multivariate time series forecating with graph neural network (MTGNN). This method allows for the consideration of complex spatio-temporal connections in multivariate time series, enhancing prediction accuracy by accounting for the complex interconnections between countries, which are not predetermined. The MTGNN outperformed traditional methods like AR, VAR-MLP, RNN-GRU, and TCN in forecasting stock prices, achieving higher accuracy. This improved predictive capability is particularly beneficial for emerging markets, which are often less stable and more sensitive to both domestic and global economic conditions. 

\end{abstract}

%%Graphical abstract
\begin{graphicalabstract}
%\includegraphics{grabs}
\end{graphicalabstract}


%%Research highlights
\begin{highlights}
\item Research highlight 1
\item Research highlight 2
\end{highlights}



\begin{keyword}
Deep learning \sep G7 countries \sep Graph neural networks \sep MINT countries \sep Multivariate time series \sep Stock price prediction

\JEL C45 \sep C53 \sep C55 \sep C82 \sep F47 \sep O50
%% keywords here, in the form: keyword \sep keyword
%% PACS codes here, in the form: \PACS code \sep code
%% MSC codes here, in the form: \MSC code \sep code
%% or \MSC[2008] code \sep code (2000 is the default)
\end{keyword}


\end{frontmatter}