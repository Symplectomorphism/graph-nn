\section{Introduction}
\label{sec:intro}
%
In the early 2010s, four countries attracted the attention of economists with
their highly promising economies, and a new acronym, MINT: Mexico, Indonesia,
Nigeria, and T\"{u}rkiye, emerged ~\citep{durotoye2014mint}. When the acronym
was first suggested by Jim O'Neill, common characteristics of MINT countries
were (i) geographical positions \textit{(Mexico sits next to the USA and belongs
to the North American Free Trade Agreement (NAFTA), Indonesia lies at the centre
of South East Asia, T\"{u}rkiye is connected to both the West and East, while
Nigeria is on the coast of Africa surrounded by future trading partners)}, (ii)
large populations \textit{(primarily under 30)}, (iii) rapid economic growth
potential, (iv) a developing middle class, and (v) high levels of
entrepreneurship ~\citep{Jim, Newland, Nagashybayeva}. Within the last decade
the economic performance of MINT countries have been mixed, with some facing
economic difficulties, including high inflation and political uncertainty that
affect the investment climate ~\citep{Okoroafor}. As further detailed
in~\cite{zhang2021stock,siddiqui2023assessing}, MINT countries as emerging
economies may not only have been affected by domestic challenges in this decade,
but may also have been influenced by global conjuncture. In particular, the
policies advanced economies such as the group of seven countries follows have 
had a huge impact.

The group of seven developed countries (G7), comprises the world's largest and
most advanced economies: Canada, France, Germany, Italy, Japan, the
United Kingdom, and the United States. They have a substantial impact on the
development of global economic policies and trends.  As of 2023, these countries may be characterized by the following properties: they
(i) have the largest share of the world's gross domestic product (GDP)
\textit{(approximately 30\% in 2023, it was more than 43\% in the early 2000s)},
(ii) are important players in global trade and investment due to
multinational corporations, and (iii) have the most liquid and developed
financial market centers that influence global capital flows and market dynamics
~\citep{World, United,
Statista}. These properties mean that G7 countries significantly impact emerging
economies such as BRICS: Brazil, Russia, India, China, South Africa and MINT by
shaping their economic policies, trade relations, and financial markets.
The stock markets of emerging economies are often highly sensitive
and closely tied to financial and economic developments in the G7 countries
because financial markets are interconnected ~\citep{fratzscher2012capital,
rey2015dilemma, acharya2020financial}. It is of utmost required that
investors and policymakers understand and interpret the links between developed
and emerging markets. Our goal in this study is to meet this need using the 
framework of spatio-temporal graph neural networks (GNN). In particular, we
employ a particular spatio-temporal GNN instance, MTGNN,  to reveal these
connections~\citep{wu2020connecting} and use them to produce very accurate 
price predictions for both the MINT and G7 stock markets.

Our study contributes to the current literature in a few important ways. First,
this study proposes to view the stock market indices of MINT and G7 countries 
as an interconnected network, modeled by an underlying graph structure that 
can be learned from historical data.
Second, this study demonstrates how spatio-temporal GNN can be used to 
effectively  capture complex and unknown interdependencies between economic
blocks, in terms of their stock market indices. Third, leveraging the graph
structure of stock indices and their interconnections, the study shows that 
prediction accuracy and robustness can be significantly improved over
traditional, hybrid, and the more-recent deep-learning approaches.

Accurate stock price forecasting is critical to support decision-making across
multiple domains, ensuring better financial outcomes and contributing to the
stability and growth of the global economy. The significance of this study can
be better appreciated by considering how it can be used to create lucrative
portfolios while taking into account the interconnectedness of the stock markets
of MINT and G7 countries. 

The rest of the paper is organized as follows: Section 2 reviews the existing
literature about methods of stock price prediction. Section 3 discusses the data
and its properties. Section 4 provides a short review of spatio-temporal GNN
methodology, in particular that of MTGNN, and Section 5 presents empirical
results. Finally, Section 6 provides the conclusions with directions for
future research.
